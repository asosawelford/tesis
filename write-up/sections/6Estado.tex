\section{ESTADO DEL ARTE}
Determinar la calidad del habla sintetizada es una problemática que atraviesa distintas áreas y tecnologías: sistemas de texto a voz (TTS), mejora del habla (Speech Enhancement), y conversión de la voz (Voice Conversion) entre otros. Para el desarrollo de estos tipos de sistemas, donde las características de la señal de salida deben ser evaluadas repetidamente, surge la necesidad de utilizar modelos de calidad automáticos basados en criterios matemáticos y psicoacústicos para poder aproximar la apreciación subjetiva humana que se obtendría, por ejemplo, con un test MOS. El estado del arte de estas técnicas esta conformado por los siguientes sistemas:

%La Distancia Cepstral de Mel (MCD por sus siglas en inglés) \cite{melcepstral} es comúnmente %utilizada para medir la calidad del habla en la tarea de conversión de voz. MCD mide la %distorsión de distintos rasgos acústicos de una señal, sin embargo, los mismos no siempre se %ven correlacionados con la percepción humana.


\subsection{PESQ y POLQA}
Dentro del campo del speech enhancement, el PESQ \cite{pesq} (Perceptual Evaluation of Speech Quality o, evaluación percibida de calidad del habla), ITU T P.862, consiste en una evaluación intrusiva para cuantificar la calidad del habla. Es un algoritmo Full Reference (FR, o referencia completa), lo que quiere decir que para realizar una evaluación sobre un sistema requiere de la señal de entrada y de salida del mismo.  Su funcionamiento parte de un modelo psicoacústico, refinado empíricamente, que estima un valor MOS comparando la referencia original con la salida degradada del modelo, usando distancias paramétricas entre ambas señales. Al comparar la señal original y la señal degradada, las alinea en tiempo y normaliza en amplitud, por lo que no tiene en cuenta los efectos de distorsión temporal y de atenuación de la señal. Sin embargo, en muchos casos, para sistemas de TTS, no contamos con las señales originales utilizadas para entrenar una red neuronal (voz original), por lo que no se puede depender de algoritmos de este tipo.

En 2011, POLQA (Perceptual Objective Listening Quality Analysis)\cite{polqa}, ITU-T P.863, fue desarrollado como sucesor a PESQ. Este algoritmo compara muestra a muestra una señal degradada por un sistema, contra un señal original tomada como entrada de dicho sistema. Se analizan ambas señales en es dominio frecuencial, en distintas bandas criticas. Las diferencias encontradas  en cada banda son consideradas distorsiones, que luego son consideradas a la hora de asignar una puntuación tipo-MOS en una escala de 1-5. El aporte más relevante de este algoritmo es su modelo perceptivo, que toma en consideración ciertos factores humanos (\textit{Idealización}) de las tareas de categorización que se realizan durante tests MOS.

\subsection{ViSQOL}
ViSQOL (Virtual Speech Quality Objective Listener o, calidad del habla objetiva virtual) \cite{visqol}, fue desarrollado para emular la percepción humana sobre la calidad del habla. Evalúa una distancia calculada sobre un \textbf{neurograma}, análogo a un espectrograma, pero cuya intensidad (variable asignada al color del gráfico) está referida a la actividad neuronal. Nuevamente se trata de un algoritmo FR. Una comparación con las métricas desarrolladas por ITU, PESQ y POLQA, se realizó teniendo en cuenta la capacidad de cada algoritmo de detectar distintos tipos de transformaciones, incluyendo el añadido de distintos tipos de ruido de fondo, filtrado de señal, mejora del habla y variación de relación señal a ruido.

Los resultados de la investigación demostraron que ViSQOL y POLQA tienen un desempeño comparable, ambos superando el algoritmo PESQ.


\subsection{MOSNet}
Desarrollado para asistir en las tareas de evaluación de conversión de voz, MOSNet\cite{mosnet} es un predictor de valores MOS. El método propuesto consiste en entrenar una red neuronal sobre una base de datos construida a partir de evaluaciones de escucha realizadas durante el Voice Conversion Challenge 2018 (VCC). Para modelar la percepción humana tres diferentes arquitecturas son puestas a prueba y comparadas a lo largo de la investigación conducida por Chen-Chou, et al. 

El primer modelo, basado en una red convolucional concatenada a una capa completamente conectada, fue derivado del trabajo previo desarrollado por Yoshimura et al. \cite{yoshimura}. Las capas convolucionales fueron configuradas empíricamente para segmentar el discurso evaluado en secciones de 400 ms a modo de capturar información temporal más corta. El segundo modelo consiste en una red BLSTM (Bidirectional Long Short-Term Memory) previamente implementada en el paper Quality-Net \cite{qualitynet}, posee la habilidad de integrar la información de dependencias en el tiempo y de características secuenciales propias de una voz humana. Finalmente el tercer modelo es diseñado como una combinación de los dos previamente mencionados. 
Para cada arquitectura propuesta, el entrenamiento se realiza sobre características espectrales extraídas del VCC, con los puntajes MOS de dicha competencia como la solución objetivo. 
Los resultados indican una correlación alta entre los puntajes MOS derivados de los modelos entrenados, y los obtenidos por medio de pruebas subjetivas.

 \subsection{NISQA}
En 2021, Mittag y Moller \cite{mittag} presentaron un evaluador de naturalidad del habla sintetizada, basada en una red neuronal CNN-LSTM obteniendo resultados satisfactorios para oraciones, con pequeñas limitaciones cuando el espectro de la onda resultante se ve acotado. 
La base de datos utilizada en el entrenamiento esta compuesta de 16 fuentes distintas extraídas de distintas competencias realizadas de forma online, divididas en 12 idiomas distintos, para desarrollar una red neuronal capaz de procesar distintos lenguajes.
Una implementación abierta del código desarrollado por esta investigación se encuentra disponible. La misma permite ser re-entrenada con una nueva base de datos.

\subsection{Sinopsis de las distintas metricas de calidad objetiva presentadas}

En la Tabla \eqref{tab:ea}, se presenta una comparación entre las distintas arquitecturas discutidas en esta sección.

\begin{table}[h]
\centering
\caption{Sinopsis de las distintos modelos de calidad propuestos para predecir la preferencia humana}
\label{tab:ea}
\begin{tabular}{llll}
\textbf{Año} & \textbf{Referenica} & \textbf{Arquitectura}                                                                 & \textbf{Comentarios}                                                                                                       \\ \hline
2001         & PESQ                & Comparación intrusiva                                                                 & \begin{tabular}[c]{@{}l@{}}Primer metodología automatizada \\ adoptada por ITU\end{tabular}                                \\
2011         & POLQA               & \begin{tabular}[c]{@{}l@{}}Comparación intrusiva con\\ modelo perceptivo\end{tabular} & \begin{tabular}[c]{@{}l@{}}Sucesor de PESQ desarrollado \\ por ITU\end{tabular}                                            \\
2015         & VISQOL              & \begin{tabular}[c]{@{}l@{}}Comparación intrusiva con\\ modelo perceptivo\end{tabular} & \begin{tabular}[c]{@{}l@{}}Introducción del Neurograma \\ como modelo perceptivo\end{tabular}                              \\
2021         & MOSNet              & CNN-BLSTM                                                                             & \begin{tabular}[c]{@{}l@{}}Red neuronal entrenada sobre resultados\\ de encuestas tipo MOS de naturalidad\end{tabular}     \\
2021         & NISQA               & CNN‐LSTM                                                                              & \begin{tabular}[c]{@{}l@{}}También calcula otros parámetros \\ acústicos (ruido, distorsión y discontinuidad)\end{tabular}
\end{tabular}
\end{table}
\newpage