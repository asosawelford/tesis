\begin{center}
    \textbf{\LARGE{RESUMEN}}
\end{center}

En esta investigación se aborda el desarrollo de un procedimiento para la determinación objetiva de la calidad de la voz humana generada por sistemas de síntesis artificiales. Se presenta la metodología adoptada para la implementación de un sistema basado en redes neuronales que sea capaz de predecir una valoración subjetiva sobre la naturalidad de una voz sintetizada. El entrenamiento y evaluación de dicho modelo fue realizado sobre una base de datos creada a partir de distintas voces sintetizadas por  algoritmos de texto a voz, grabaciones de discurso humano reales, y grabaciones procesadas digitalmente de ambos grupos previamente mencionados. Dichas voces fueron juzgadas subjetivamente en un test tipo-MOS realizado de forma online por \color{red} [\textit{Completar con el numero de participantes de la encuesta subjetiva}] \color{black}. A partir de los resultados obtenidos se observa \color{red} [\textit{Completar con los resultados obtenidos (correlación de la metrica obtenida respecto de las evaluaciones subjetivas, y añadir conclusiones mas relevantes}] \color{black}
\\
\\
\footnotesize
\textbf{\textit{Palabras clave: calidad del habla, texto a voz, evaluación objetiva, deep learning }}
\normalsize

\newpage